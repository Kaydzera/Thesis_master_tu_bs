\chapter{Implementation Details}
\label{sec:implementation}

% TODO: Write implementation section

\section{Software Architecture}

The implementation consists of several modular components:

\begin{itemize}
    \item \textbf{models.py:} Data structures for problems and search nodes
    \item \textbf{bnb.py:} Branch-and-bound solver with pruning logic
    \item \textbf{knapsack\_dp.py:} Dynamic programming solver for ceiling knapsack bounds
    \item \textbf{solvers.py:} List scheduling and IP formulations for follower
    \item \textbf{bilevel\_gurobi.py:} Complete enumeration baseline
    \item \textbf{logger.py:} Comprehensive logging and metrics collection
\end{itemize}

\section{Key Data Structures}

\subsection{Problem Node}

Each node in the search tree maintains:

\begin{lstlisting}
class ProblemNode:
    job_occurrences: List[int]  # Quantities of each job type
    depth: int                  # How many items decided
    remaining_budget: float     # Budget left after decisions
    _max_at_depth: float       # Cached contribution from decided items
    _branch_type: str          # 'root', 'left', 'right', 'lower'
\end{lstlisting}

The cached value \texttt{\_max\_at\_depth} stores $\sum_{i=1}^{d} d_i \lceil x_i / m \rceil$ to avoid recomputation.

\subsection{Ceiling Knapsack Solver}

Pre-computes DP table for efficient bound queries:

\begin{lstlisting}
class CeilKnapsackSolver:
    def __init__(self, costs, durations, m, max_budget):
        # Build DP table: dp[i][b] = best value for items 0..i, budget b
        # Items stored in REVERSED order for querying last k items
        
    def query(self, num_items, budget):
        # Return best value using first num_items (last in original order)
        return self.dp[num_items][budget]
\end{lstlisting}

\section{Algorithmic Optimizations}

\subsection{1. Depth-First Search}

We use a depth-first search strategy to explore the branch-and-bound tree:
\begin{itemize}
    \item Start with the heuristic solution to initialize the incumbent
    \item Generate adjacent nodes (children) from the root
    \item Prioritize lower children (nodes at greater depth) in the search order
    \item Explore deeply into the tree before backtracking
    \item Memory-efficient compared to breadth-first approaches
\end{itemize}

\subsection{2. Heuristic Initialization}

We compute an initial incumbent solution in two stages:
\begin{enumerate}
    \item Solve a standard knapsack problem to maximize the sum of durations $\sum_i d_i x_i$ subject to the budget constraint $\sum_i p_i x_i \leq B$, without considering the scheduling aspect
    \item Pass the resulting job selection to the scheduler, which solves the follower's scheduling problem optimally via integer programming
\end{enumerate}
This approach provides a strong starting solution by selecting jobs with high total duration, which the scheduler then arranges to achieve a large makespan.

\subsection{3. Incremental Budget Tracking}

Each node tracks remaining budget to avoid recalculation:
\begin{itemize}
    \item Parent budget - (quantity × price)
    \item Constant-time budget feasibility checks
\end{itemize}

\subsection{4. Cached Contribution Values}

The makespan contribution $C_{\max}^{\text{LPT},(d)}(x_1, \ldots, x_d)$ from decided items is computed incrementally:
\begin{itemize}
    \item When generating a child node at depth $d+1$, we update the machine loads by adding the $x_{d+1}$ jobs of duration $d_{d+1}$ using the LPT rule
    \item This incremental update avoids recomputing the entire schedule from scratch at each node
    \item Reduces complexity from $O(d \cdot J)$ to $O(x_{d+1} \log m)$ per node expansion
\end{itemize}

\section{Logging and Metrics}

Comprehensive logging tracks:
\begin{itemize}
    \item Node visits and evaluations
    \item Incumbent updates
    \item Pruning events (by reason)
    \item Bound computations
    \item Runtime statistics
\end{itemize}

Example log entry (JSON format):
\begin{lstlisting}
{
  "event": "node_pruned",
  "reason": "bound_dominated",
  "depth": 5,
  "bound": 42.0,
  "incumbent": 45.0,
  "timestamp": 1.234
}
\end{lstlisting}

\section{Testing and Verification}

\subsection{Correctness Verification}

For small instances, we verify branch-and-bound against complete enumeration:
\begin{itemize}
    \item 10 small instances (4-5 jobs, 2-4 machines)
    \item All solutions match enumeration exactly
    \item 100\% verification rate
\end{itemize}

\subsection{Bound Tightness Testing}

We validate that upper bounds are indeed optimistic:
\begin{itemize}
    \item For every evaluated node: $\text{UB}(x) \geq C_{\max}(x)$
    \item No false pruning detected
\end{itemize}

\subsection{Performance Benchmarking}

We compare against:
\begin{itemize}
    \item Complete enumeration (for small instances)
    \item Gurobi MIP solver for scheduling subproblems
    \item Pure knapsack heuristics
\end{itemize}

\section{Code Example: Bound Computation}

\lstinputlisting[
    caption={Excerpt from bound computation code},
    label={lst:bound-code},
    firstline=1,
    lastline=20
]{code/bound_computation_example.py}
