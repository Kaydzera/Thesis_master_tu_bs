\chapter{Conclusion and Future Work}
\label{sec:conclusion}

% TODO: Write conclusion

\section{Summary of Contributions}

This thesis addressed the bilevel knapsack-scheduling optimization problem, where a leader selects job types within a budget to maximize system capacity, anticipating that a follower will schedule jobs to minimize makespan.

\textbf{Key achievements:}
\begin{enumerate}
    \item \textbf{Efficient algorithm:} Developed a branch-and-bound algorithm achieving speedups of up to 188× compared to complete enumeration
    
    \item \textbf{Tight bounds:} Designed a knapsack scheduling relaxation providing upper bounds that enable effective pruning (average 37\% pruning rate)
    
    \item \textbf{Practical scalability:} Demonstrated ability to solve instances with 10 job types and 10 machines in seconds
    
    \item \textbf{Comprehensive evaluation:} Tested on 130 instances ranging from small (verified optimal) to large (stress tests)
    
    \item \textbf{Implementation:} Created modular, well-documented Python implementation with extensive logging
\end{enumerate}

\section{Key Insights}

\textbf{1. Importance of tight bounds:}
The knapsack scheduling relaxation provides bounds that are both:
\begin{itemize}
    \item Optimistic (never underestimate true optimal value)
    \item Tight enough to prune most of the search space
    \item Computable efficiently via dynamic programming
\end{itemize}

\textbf{2. Search space structure:}
Problem instances exhibit significant variability:
\begin{itemize}
    \item Uniform-price items: easier to prune
    \item Wide variety in ratios: more exploration needed
    \item Budget tightness affects pruning effectiveness
\end{itemize}

\textbf{3. Enumeration limitations:}
Complete enumeration becomes infeasible rapidly:
\begin{itemize}
    \item Instance \#14: 3.3 billion combinations
    \item Enumeration ran 2 hours without finding optimum
    \item BnB found optimal in 2.43 seconds
\end{itemize}

\section{Limitations}

\textbf{1. Upper bound approximation:}
\begin{itemize}
    \item The upper bound computation uses the LPT rule (a $\frac{4}{3}$-approximation) for the $C_{\max}^{\text{LPT},(d)}$ term
    \item This may slightly overestimate the true contribution from decided items
    \item However, the follower's scheduling problem is solved optimally via integer programming (Gurobi)
    \item The approximation only affects bound tightness, not solution optimality
\end{itemize}

\textbf{2. Scalability bounds:}
\begin{itemize}
    \item Instances with 12+ job types become challenging
    \item Node limit (500k) occasionally hit on large instances
    \item Exponential worst-case complexity
\end{itemize}

\textbf{3. Single-objective:}
\begin{itemize}
    \item Only considers makespan
    \item Real applications may have multiple objectives (cost, energy, fairness)
\end{itemize}

\section{Future Research Directions}

\subsection{1. Algorithmic Improvements}

\textbf{Enhanced pruning:}
\begin{itemize}
    \item Implement optimality dominance (detect when bound is achieved)
    \item Symmetry breaking for identical items
    \item Learning-based node selection
\end{itemize}

\textbf{Parallel search:}
\begin{itemize}
    \item Distribute subtree exploration across multiple cores
    \item Work-stealing for load balancing
    \item Shared incumbent for global pruning
\end{itemize}

\textbf{Hybrid approaches:}
\begin{itemize}
    \item Combine with local search for large neighborhoods
    \item Column generation for many item types
    \item Constraint programming integration
\end{itemize}

\subsection{2. Problem Extensions}

\textbf{Generalized scheduling:}
\begin{itemize}
    \item Unrelated parallel machines (different speeds)
    \item Precedence constraints between jobs
    \item Machine eligibility restrictions
\end{itemize}

\textbf{Stochastic version:}
\begin{itemize}
    \item Uncertain processing times
    \item Probabilistic budget constraints
    \item Robust optimization formulations
\end{itemize}

\textbf{Multi-objective:}
\begin{itemize}
    \item Pareto frontier exploration
    \item Cost vs. capacity trade-offs
    \item Energy-aware scheduling
\end{itemize}

\subsection{3. Applications}

\textbf{Production scheduling:}
\begin{itemize}
    \item Job type selection for manufacturing facilities
    \item Capacity planning with scheduling constraints
    \item Resource allocation under processing time uncertainty
\end{itemize}

\textbf{Manufacturing:}
\begin{itemize}
    \item Tool selection for job shops
    \item Resource allocation in production planning
    \item Maintenance scheduling with capacity constraints
\end{itemize}

\textbf{Project management:}
\begin{itemize}
    \item Resource procurement for project portfolios
    \item Budget allocation across multiple projects
    \item Workforce planning with skill diversity
\end{itemize}

\section{Closing Remarks}

Bilevel optimization problems arise naturally in many hierarchical decision-making contexts. This thesis demonstrates that sophisticated algorithms combining intelligent search, tight bounding, and effective pruning can solve practical instances efficiently.

The branch-and-bound framework provides a principled approach that guarantees optimality while achieving orders-of-magnitude speedups over naive enumeration. The success on real-world-sized instances (10 job types, 10 machines) suggests the methodology is viable for deployment in operational systems.

Future work should focus on scaling to even larger instances through parallelization and exploring richer problem variants that capture additional real-world complexities. The foundation established here provides a solid platform for such extensions.
