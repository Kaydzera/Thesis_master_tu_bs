\chapter{Problem Formulation}
\label{sec:formulation}

% TODO: Write problem formulation

\section{Bilevel Optimization Framework}

A bilevel optimization problem has a hierarchical structure with two decision-makers:

\begin{definition}[Bilevel Optimization Problem]
\label{def:bilevel}
\begin{align}
\max_{x} \quad & F(x, y^*(x)) \label{eq:bilevel-upper}\\
\text{s.t.} \quad & x \in X \nonumber\\
\text{where } y^*(x) \in & \arg\min_{y} \{f(x, y) : y \in Y(x)\} \label{eq:bilevel-lower}
\end{align}
\end{definition}

\begin{note}
This general framework encompasses a wide range of hierarchical optimization problems. Additional theoretical aspects, such as solution existence, uniqueness conditions, and computational complexity classes, may be discussed in future extensions of this work.
\end{note}

The leader chooses $x$ to optimize their objective $F$, anticipating that the follower will respond by solving their own optimization problem to find $y^*(x)$.

\section{The Knapsack-Scheduling Problem}

\subsection{Problem Data}

\begin{itemize}
    \item $n$ job types, indexed by $i = 1, \ldots, n$
    \item Job type $i$ has:
    \begin{itemize}
        \item Processing time (duration): $d_i > 0$
        \item Cost (price): $p_i > 0$
    \end{itemize}
    \item $m$ identical parallel machines
    \item Total budget: $B > 0$
\end{itemize}

\subsection{Leader's Problem (Upper Level)}

The leader selects how many jobs of each type to purchase:

\textbf{Decision variables:} $x_i \in \mathbb{Z}_+$ = number of jobs of type $i$ to select

\textbf{Constraints:}
\begin{equation}
\sum_{i=1}^{n} p_i x_i \leq B \quad \text{(budget constraint)}
\end{equation}

\textbf{Objective:} Maximize the makespan resulting from optimal scheduling:
\begin{equation}
\max_{x} \quad C_{\max}^*(x)
\end{equation}
where $C_{\max}^*(x)$ is the optimal makespan for the follower's problem given selection $x$.

\subsection{Follower's Problem (Lower Level)}

Given the leader's selection $x = (x_1, \ldots, x_n)$, the follower has a total of $\sum_{i=1}^{n} x_i$ jobs to schedule.

\textbf{Decision variables:} For each job $j$, assign it to some machine

\textbf{Objective:} Minimize makespan (maximum machine load)
\begin{equation}
C_{\max}^*(x) = \min \max_{k=1,\ldots,m} L_k
\end{equation}
where $L_k$ denotes the load on machine $k$, defined as the total processing time of all jobs assigned to machine $k$ after scheduling.

\section{Complete Bilevel Formulation}

\begin{align}
\max_{x \in \mathbb{Z}_+^n} \quad & C_{\max}^*(x) \label{eq:leader-obj}\\
\text{s.t.} \quad & \sum_{i=1}^{n} p_i x_i \leq B \label{eq:budget}\\
\text{where } C_{\max}^*(x) = \quad & \min_{y} \max_{k=1}^{m} \sum_{i=1}^{n} \sum_{j=1}^{x_i} d_i \cdot y_{ijk} \label{eq:follower-obj}\\
\text{s.t.} \quad & \sum_{k=1}^{m} y_{ijk} = 1 \quad \forall i, j \label{eq:assign}\\
& y_{ijk} \in \{0, 1\} \quad \forall i, j, k \nonumber
\end{align}

Here, $y_{ijk} = 1$ if copy $j$ of job type $i$ is assigned to machine $k$.

\section{Problem Characteristics}

\begin{itemize}
    \item \textbf{Complexity:} The problem is NP-hard, combining two NP-hard problems (knapsack and scheduling)
    \item \textbf{Non-convexity:} The follower's optimal value function $C_{\max}^*(x)$ is non-convex and discontinuous
    \item \textbf{Discrete bilevel:} Cannot use KKT conditions (only valid for continuous problems)
    \item \textbf{Enumerative approach needed:} Must explore discrete solution space
\end{itemize}

\section{Interpretation: Why Maximize Makespan?}

The leader maximizes makespan to ensure robust system capacity. This models scenarios where:
\begin{itemize}
    \item The leader wants to stresstest the system capacity
    \item The follower (scheduler) will always try to minimize completion time by finding an optimal schedule
    \item Models adversarial or worst-case planning
\end{itemize}

\textbf{Example:} A production manager selects job types (leader) knowing that a scheduler will minimize completion time (follower). The manager wants to maximize the workload the system can handle.
